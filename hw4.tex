\documentclass[11pt]{article}

% To produce a letter size output. Otherwise will be A4 size.
\usepackage[letterpaper]{geometry}

% For enumerated lists using letters: a. b. etc.
\usepackage{enumitem}

\topmargin -.5in
\textheight 9in
\oddsidemargin -.25in
\evensidemargin -.25in
\textwidth 7in

\begin{document}

% Edit the following putting your first and last names and your lab section.
\author{Nabir Migadde\\
Lab CSE-015-04L F 4:30-7:20pm}

% Edit the following replacing X with the HW number.
\title{CSE 015: Discrete Mathematics\\
Spring 2020\\
Homework \#4\\}

% Put today's date in the following.
\date{March 8, 2020}
\maketitle
\begin{enumerate}

\item
The sum of two odd integers is even.

Even integers = 2n

Odd integers = 2n+1

Sum of two odd integers = (2n+1) + (2n+1) = 4n+2 = 2(n+1)

It is multiplied by 2, so it can be divided by 2 without any remainders. So it is even.

\item
The sum of two even integers is even.

Even integers = 2n

Sum of two even integers = 2n + 2n = 4n = 2(2n)

It is multiplied by 2, so it can be divided by 2 without any remainders. So it is even.

\item
The square of an even number is even.

Even number = 2n

Square of even number = (2n)(2n) = $4n^2$ = 2($2n^2$)

It is multiplied by 2, so it can be divided by 2 without any remainders. So it is even.

\item
The product of two odd integers is odd.

Odd integers = 2n+1

Product of two odd integers = (2n+1)(2n+1) = $4n^2+4n+1$ = 2($2n^2+2n)+1$

It is multiplied by 2, so it can be divided by 2, but there will always be a remainder. So it is odd.

\item
If $n^3+5$ is odd then $n$ is even, for any $n \in \mathbb{Z}$.

If n is even then n = 2a = $(2a)^3+5$ = $8a^3+4+1$ = $2(4a^3+2)+1$

Which is multiplied by 2, so it can be divided by 2, but there will always be a remainder due to the +1. So $n^3+5$ is odd.

If n is odd then n = 2a+1 = $(2a+1)^3+5$ = $8a^3+1+3*4a^2+3*2a+5$ = $2(4a^3+6a^2+3a+3)$

then $n^3+5$ would have to be even because the end result is multiplied by 2, so it can be divided by 2 without any remainders.

\item
If $3n+2$ is even then $n$ is even, for any $n \in \mathbb{Z}$.

If n is even then n = 2a = $3(2a)+2$ = $6a+2$ = $2(3a+1)$

Which is multiplied by 2, so it can be divided by 2 without any remainders. So $3n+2$ is even.

If n is odd then n = 2a+1 = $3(2a+1)+2$ = $6a+5$ = $2(3a+2)+1$

then $3n+2$ would have to be even because the end result is multiplied by 2, so when it is divided by 2 there will always be a remainder.

\item
 The sum of a rational number and an irrational number is irrational.
 
 Assume: rational + irrational = rational
 
 rational = a/b form
 
 irrational = not in a/b form
 
 a/b + x = m/n
 
 x = m/n - a/b
 
 = mb-an/nb = it is in p/q form
 
 but x is irrational(not in p/q form) so we assumed wrong.
 
 Therefore rational + irrational = irrational.
 
 \item
 The product of two irrational numbers is irrational.
 
 Two irrational numbers = (a,b)
 
 a*b=c
 
 1/n * n = 1 $\rightarrow$ rational
 
 a*b=c
 
 n*n = $n^2 \rightarrow$ irrational
 
 There sometimes the product of two irrational numbers is irrational and other times there are rational. It depends on the a,b.
 
 \item
 $P(n): n^3+2n$ is divisible by 3, for any integer greater than 0.
 
 for n = 0, 0+0 is divisible by 3, so P(0) is true
 
 for n = 1, $1^3+2(1) = 3$ is divisible by 3, so P(1) is true
 
 let P(n) be true for n=m, $m^3+2m$ divisible by 3.
 
 then $m^3+2m = 3k$ for some integer k
 
 Now, $(m+1)^3+2(m+1)= m^3+3m^2+3m+1+2m+2$
 
 $=m^3+2m+3m^2+3m+3$
 
 $=3k+3m^2+3m+3$ [since $m^3+2m=3k]$
 
 $=3(m^2+m+k+1)$
 
$(m+1)^3+2(m+1)$ is also divisible by 3

Therefore, P(n) true for n=0,1,m+1 whenever P(m) true. So by mathematical induction $P(n): n^3+2n$ true for all n $\geqslant$ 0.

 \item
 $P(n): n^3+2n$ is divisible by 3, for any integer greater than 0.
 
 for n = 0, 0+0 is divisible by 3, so P(0) is true
 
 for n = 1, $1^3+2(1) = 3$ is divisible by 3, so P(1) is true
 
 let P(n) be true for n=m, $m^3+2m$ divisible by 3.
 
 then $m^3+2m = 3k$ for some integer k
 
 Now, $(m+1)^3+2(m+1)= m^3+3m^2+3m+1+2m+2$
 
 $=m^3+2m+3m^2+3m+3$
 
 $=3k+3m^2+3m+3$ [since $m^3+2m=3k]$
 
 $=3(m^2+m+k+1)$
 
$(m+1)^3+2(m+1)$ is also divisible by 3

Therefore, P(n) true for n=0,1,m+1 whenever P(m) true. So by mathematical induction $P(n): n^3+2n$ true for all n $\geqslant$ 0.



\end{enumerate}

\end{document}