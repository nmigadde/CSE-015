\documentclass[11pt]{article}

% To produce a letter size output. Otherwise will be A4 size.
\usepackage[letterpaper]{geometry}

% For enumerated lists using letters: a. b. etc.
\usepackage{enumitem}

\topmargin -.5in
\textheight 9in
\oddsidemargin -.25in
\evensidemargin -.25in
\textwidth 7in

\begin{document}

% Edit the following putting your first and last names and your lab section.
\author{Nabir Migadde\\
Lab CSE-015-04L F 4:30-7:20pm}

% Edit the following replacing X with the HW number.
\title{CSE 015: Discrete Mathematics\\
Spring 2020\\
Homework \#3\\}

% Put today's date in the following.
\date{February 28, 2020}
\maketitle

Knights and Knaves

\begin{enumerate}

\begin{center}
\begin{tabular}{|c|c|c|}
\hline
Case & P & Q \\
\hline
1 & Knight & Knight \\
2 & Knave & Knight \\
3 & Knight & Knave \\
4 & Knave & Knave \\
\hline
\end{tabular}
\end{center}

\item
Knave: always tells lie

Knight: always tells truth

Looking at the truth table, it can be seen that P's statement is false for case 1 and True for cases 2,3, and 4. If P is a Knight then what he says would be true, that means case 1 can't happen but case 3 can. If P is a Knave, whatever he would say must be False, meaning neither case 2 nor case 4 can happen as Knave always lie. Thus the only possible case is Row 3 and therefore P is a Knight and Q is a Knave.

\begin{center}
\begin{tabular}{|c|c|c|}
\hline
Case & A & B \\
\hline
1 & Knight & Knight \\
2 & Knave & Knight \\
3 & Knight & Knave \\
4 & Knave & Knave \\
\hline
\end{tabular}
\end{center}

\item
Knave: always tells lie

Knight: always tells truth

Its the same thing, because its the same truth table. So, A's statement is false for the case 1 and True for the cases 2,3, and 4. If A is a Knight then what he says would be true, that means case 1 and case 3 can't happen. If A is a knave then what he says would be False, this means case 4 can't happen. The only possibility is that A is a Knave and B is a Knight, as in, case 2 is the only possible scenario.

\end{enumerate}


Logical Identities

\begin{enumerate}

\item
$\lnot(p \rightarrow (q \rightarrow p))$

Implication law, $q \rightarrow p \equiv \lnot q \lor p$

$\lnot(p \rightarrow (\lnot q \lor p))$

Implication law 
$\lnot(\lnot p \lor (\lnot q \lor p))$

De Morgan's Law
$\lnot(\lnot p) \land \lnot (\lnot q \lor p)$

De Morgan's Law
$\lnot(\lnot p) \land \lnot (\lnot q) \land \lnot p$

Double Negation Law 
$\lnot(\lnot p) \equiv p$

$p \land q \land \lnot p$

$p \land \lnot p \land q$

$p \land \lnot \equiv False$ , thus

False $\land q$

$\equiv q$

\item
$\lnot((p \land q) \rightarrow (q \lor p))$

Let $(p \land q)$ be a, $(q \lor p)$ be b

Implication Law, $p \rightarrow q \equiv \lnot p \lor q$

$\lnot a \lor b$

which is $\lnot(p \land q) \lor (q \lor p)$

De Morgan's Law, $\lnot (p \land q) \equiv \lnot p \lor \lnot q$

$\lnot p \lor \lnot q \lor (q \lor p)$

Associative Law, 
$(\lnot p \lor p) \lor (\lnot q \lor q)$

Negation Laws, $p \lor \lnot p \equiv True$

True $\lor$ True

$\equiv True$

\end{enumerate}

Logical Equivalences
\begin{enumerate}

\item
$p \rightarrow (q \rightarrow r)$ and $(p \land q) \rightarrow r$

\begin{center}
\begin{tabular}{|c|c|c|c|c|}
\hline
$p$ & $q$ & $r$ & $q \rightarrow r$ & $p \rightarrow (q \rightarrow r)$\\
\hline
F & F & F & T & T\\
F & F & T & T & T\\
F & T & F & F & T\\
F & T & T & T & T\\
T & F & F & T & T\\
T & F & T & T & T\\
T & T & F & F & F\\
T & T & T & T & T\\
\hline
\end{tabular}
\end{center}

\begin{center}
\begin{tabular}{|c|c|c|c|c|}
\hline
$p$ & $q$ & $r$ & $p \land q$ & $(p \land q) \rightarrow r$\\
\hline
F & F & F & T & T\\
F & F & T & T & T\\
F & T & F & F & T\\
F & T & T & T & T\\
T & F & F & T & T\\
T & F & T & T & T\\
T & T & F & F & F\\
T & T & T & T & T\\
\hline
\end{tabular}
\end{center}

The truth tables are the same. Therefore the propositions are equivalent.

\item
$p \rightarrow (q \rightarrow r)$ and $(p \rightarrow q) \rightarrow r$

\begin{center}
\begin{tabular}{|c|c|c|c|c|}
\hline
$p$ & $q$ & $r$ & $q \rightarrow r$ & $p \rightarrow (q \rightarrow r)$\\
\hline
F & F & F & T & T\\
F & F & T & T & T\\
F & T & F & F & T\\
F & T & T & T & T\\
T & F & F & T & T\\
T & F & T & T & T\\
T & T & F & F & F\\
T & T & T & T & T\\
\hline
\end{tabular}
\end{center}

\begin{center}
\begin{tabular}{|c|c|c|c|c|}
\hline
$p$ & $q$ & $r$ & $p \rightarrow q$ & $(p \rightarrow q) \rightarrow r$\\
\hline
F & F & F & T & F\\
F & F & T & T & T\\
F & T & F & T & F\\
F & T & T & T & T\\
T & F & F & F & T\\
T & F & T & F & T\\
T & T & F & T & F\\
T & T & T & T & T\\
\hline
\end{tabular}
\end{center}

The truth tables are not the same. Therefore the propositions are not equivalent.

\end{enumerate}

Logical Consequences

\begin{enumerate}

\item
Jimmy is smart

If Jimmy then smart

Jimmy $\rightarrow$ smart

Smart people are rich

If smart then rich

Smart $\rightarrow$ rich

Therefore, Jimmy $\rightarrow$ rich because of transitivity. Jimmy is rich.

\item
Islands are surrounded by water

If island then surrounded by water

Island $\rightarrow$ surrounded by water

Puerto Rico is surrounded by water

Puerto Rico $\rightarrow$ surrounded by water

Puerto Rico could not always be an island. Therefore we cannot say Puerto Rico is an island is always True. 


\end{enumerate}

\end{document}